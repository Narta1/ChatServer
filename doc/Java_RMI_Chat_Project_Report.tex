
\documentclass[12pt]{article}
\usepackage[a4paper, margin=1in]{geometry}
\usepackage{hyperref}
\usepackage{listings}

\title{Project Report: Java RMI Chat Application}
\author{Narta Matteo}
\date{\today}

\begin{document}

\maketitle

\section{Utilisation}
Clone the github : git@github.com:Narta1/ChatServer.git.
Then go in code directory, compile with "javac *.java". Start the server using "java Server" and the clients using "java Client username" or "java ClientGUI username" depending
if you want the graphical user interface or not.

\section{Introduction}

This project implements a distributed chat application using Java RMI (Remote Method Invocation). 
The system allows multiple clients to connect to a central server, exchange messages, and maintain 
a shared chat history. The architecture follows a client-server model where remote objects are 
used to enable communication between distributed components.

\section{System Architecture}

The application is divided into the following main components:

\begin{itemize}
    \item \textbf{Chat Interface (Chat.java)}: Defines the remote methods that clients can invoke on the server. 
    These include joining the chat, leaving, writing messages, reading messages, and saving the chat history.
    
    \item \textbf{Chat Implementation (ChatImpl.java)}: Implements the remote interface and extends 
    \texttt{UnicastRemoteObject}. It manages connected users, message history, synchronization, 
    and persistence of chat data.
    
    \item \textbf{Server (Server.java)}: Starts the RMI registry and binds the chat service under 
    a specific name (e.g., ``chatServer''). This allows clients to look up and access the remote object.
    
    \item \textbf{Client and GUI (Client.java, ClientGUI.java)}: The client connects to the RMI registry, 
    retrieves the remote chat object, and interacts with it. The graphical interface provides 
    user-friendly interaction for sending and receiving messages.
    
    \item \textbf{ChatClient Interface and Implementation}: Allows the server to call back clients 
    (callback mechanism), enabling real-time message updates.
\end{itemize}

\section{Main Functionalities}

\subsection{User Management}

Users can join and leave the chat using the \texttt{join()} and \texttt{leave()} methods. 
The server maintains a \texttt{HashMap} of connected participants, mapping usernames 
to their corresponding remote client objects.

\subsection{Messaging}

When a user sends a message using \texttt{write()}, the server stores it in a shared 
message history (a \texttt{List<String>}). Clients can retrieve messages using the 
\texttt{read()} method. The callback mechanism allows the server to notify clients 
when new messages are available.

\subsection{Concurrency Control}

Since multiple clients may access the server simultaneously, synchronization mechanisms 
are used. A \texttt{ReentrantLock} ensures thread-safe access to shared resources such 
as the participants list and message history.

\subsection{Persistence}

The chat history is saved to a file, allowing messages to be restored when the server 
restarts. This ensures that previous conversations are not lost.

\section{Technologies Used}

\begin{itemize}
    \item Java RMI for distributed communication
    \item Java Collections Framework (HashMap, ArrayList)
    \item Concurrency utilities (ReentrantLock)
    \item Java Swing for graphical user interface
\end{itemize}

\section{Conclusion}

This project demonstrates the implementation of a distributed system using Java RMI. 
It covers key distributed system concepts such as remote interfaces, object binding 
in a registry, client-server communication, synchronization, and persistence. 
The use of callbacks enhances real-time interaction, making the chat application 
efficient and responsive.

\end{document}
